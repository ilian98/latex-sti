\documentclass[12pt]{article}

%---DOCUMENT MARGINS---
\usepackage{geometry} % Required for adjusting page dimensions and margins
\geometry{
	paper=a4paper, % Paper size, change to letterpaper for US letter size
	top=4.25cm, % Top margin
	bottom=\bottomMargin, % Bottom margin
	left=2.5cm, % Left margin
	right=2.5cm, % Right margin
	headheight=3cm, % Header height
	%footskip=1.5cm, % Space from the bottom margin to the baseline of the footer
	headsep=0.8cm, % Space from the top margin to the baseline of the header
	%showframe, % Uncomment to show how the type block is set on the page
}
\usepackage{cmap}

\usepackage{fontspec}
\defaultfontfeatures{Renderer=Basic,Ligatures={TeX}}
\setmainfont{Times New Roman}
\setsansfont{Times New Roman}
\setmonofont{\monofont}
\usepackage[bulgarian]{babel}
\usepackage[math-style = TeX]{unicode-math}
\setmathfont{\mathfont}

\usepackage[nobottomtitles*]{titlesec}
\usepackage[dvipsnames]{xcolor}
\titleformat
{\section} % command
{\normalfont\fontsize{14}{14}\sffamily\bfseries} % format
{} % label
{0pt} % sep
{\hspace{-0pt}{\color{cyan}\rule{0.7\textwidth}{\parskip}}\\*[2pt]} % before-code
\titlespacing{\section}{0pt}{0em}{0em}
\titleformat
{\subsection} % command
{\fontsize{14}{14}\bfseries} % format
{} % label
{0pt} % sep
{\hspace{-24pt}{\color{cyan}\rule{0.5\textwidth}{0.5\parskip}}\\*[5pt]} % before-code
\titlespacing{\subsection}{\parindent}{0em}{0em}

\setlength{\parskip}{0.5em}
\setlength{\parindent}{24pt}
\sloppy

\usepackage{graphicx}
\graphicspath{{./images/}}
\usepackage[export]{adjustbox}
\usepackage{wrapfig}
\makeatletter
\patchcmd\WF@putfigmaybe{\lower\intextsep}{}{}{\fail}
\AddToHook{env/wrapfigure/begin}{\setlength{\intextsep}{0pt}}
\makeatother
\usepackage[inkscapearea=page,inkscapepath=./svg-inkscape]{svg}
\svgpath{{./images/}}

\usepackage{fancyhdr}
\pagestyle{fancy}
\usepackage{background}
\backgroundsetup{
	contents={\includegraphics[width=28cm,height=3.45cm]{./structure/header_background.png}},scale=1,placement=top,opacity=0.8
}
\fancyhead[L]{
	\begin{minipage}{\textwidth}
		\includegraphics[width=2.8cm]{./structure/ruo.png}
		\vspace{0.35cm}
	\end{minipage}
}
\fancyhead[R]{
	\begin{minipage}{\textwidth}
		\flushright\includegraphics[width=2.8cm]{./structure/sap.png}
		\vspace{0.35cm}
	\end{minipage}
}
\newfontfamily\headerfont[
	ItalicFont = *i,
	BoldItalicFont = *bi
]{Arial}
\fancyhead[C]{
	\begin{minipage}{\textwidth}
		\color{white}\centering\small\textbf{
			\headerfont
			\textit{ОТКРИТО ПЪРВЕНСТВО НА СОФИЯ ПО ИНФОРМАТИКА}\\
			\textit{\contestDate}\\
			\textit{Група \group, \grade\ клас}
		}
	\end{minipage}
}
\renewcommand{\headrulewidth}{0pt}
\fancyheadoffset[L,R]{1.5cm}

\raggedbottom

\usepackage{amsmath}
\usepackage{stmaryrd}
\usepackage{tabularray}
\AtBeginEnvironment{table}{\vspace{-0.2cm}}
\AtEndEnvironment{table}{\vspace{-0.2cm}}

\usepackage{placeins}
\usepackage{caption}
\captionsetup[table]{
	skip=3pt,font=it,
	singlelinecheck=false,justification=justified,indention=-24pt,
	margin={24pt, 0pt}
}

\usepackage{enumitem}
\setlist{itemsep=-0.4em,leftmargin=\parindent,topsep=-\parskip}
\newcommand{\tabitem}{\indent~~\llap{\textbullet}~~}

\usepackage{hyperref}
\hypersetup{
	colorlinks=true,
	citecolor=blue,
	linkcolor=blue,
	urlcolor=cyan,
}

\usepackage{emoji}

\begin{document}

\section*{Задача A3. ИЛИ/И}

Като отявлен любител на побитовите операции Пешо намира следното предизвикателство в социалните мрежи. Даден е един масив и се пита дали може да се разделят елементите му на две непразни множества, така че \textit{побитовото или} на елементите от първото множество да е равно на \textit{побитовото и} на елементите на второто множество. Такъв масив Пешо нарича побитов. Той бързо се справя с тази закачка и даже измисля усложнение. Нека имаме масив $a$ с $N$ елемента - $a_1, a_2, ..., a_N$. Трябва да се отговори дали са побитови $Q$ негови подмасива, всеки зададен с два индекса $l$ и $r$ - $a_l, a_{l+1}, ..., a_r$. Помогнете на Пешо да реши тази задача, като напишете програма \textbf{orand}, която намира отговорите на въпросите.

\subsection*{Вход}

От първия ред на стандартния вход се въвеждат целите числа $N$ и $Q$. От следващия ред се въвеждат $N$ числа - елементите на масива $a$. От последните $Q$ реда се въвеждат по две цели числа $l$ и $r$ - левият и десният край на подмасива за съответния въпрос.

\subsection*{Изход}
За всяка заявка, по реда във входа, изведете "Yes" или "No" (без кавичките) в зависимост от това дали подмасивът е побитов.

\subsection*{Ограничения}
\begin{itemize}
	\item $1\leq N, Q\leq 10^5$
	\item $0\leq a_i < 2^{30}$
\end{itemize}

\subsection*{Подзадачи}
\begin{table}[hbtp]
	\centering
	\begin{tblr}{|X[15,c,m]|X[9,c,m]|X[18,c,m]|X[11,c,m]|X[47,c,m]|}
		\hline
		\textbf{Подзадача} & \textbf{Точки} & \textbf{Необходими подзадачи} &
		$N$ & 
		\textbf{Допълнителни ограничения} \\
		\hline
		1 & 0 & -- & $-$ & Примерът. \\ 
		\hline
		2 & 11 & $1$ & $\leq 10^1$ & -- \\ 
		\hline
		3 & 30 & $1-2$ & $\leq 10^2$ & -- \\
		\hline
		4 & 38 & $1-3$ & $\leq 10^3$ & -- \\
		\hline
		5 & 10 & $1-4$ & $\leq 2.10^4$ & -- \\
		\hline
		6 & 11 & $1-5$ & $\leq 10^5$ & -- \\
		\hline
	\end{tblr}
\caption*{\indent Точките за дадена подзадача се получават само ако се преминат успешно всички тестове, предвидени за нея и необходимите предишни подзадачи.}
\end{table}
\FloatBarrier

\subsection*{Пример}
\begin{table}[ht]
	\centering
	\begin{tblr}{|X[2,l]|X[l]|X[6,j]|}
		\hline
		\textbf{Вход} & \textbf{Изход} & \textbf{Обяснение на примера} \\
		\hline
		{\texttt{6 4} \\
			\texttt{13 15 13 5 1 7} \\ 
			\texttt{1 5} \\ 
			\texttt{2 5} \\ 
			\texttt{3 6} \\
			\texttt{4 6}} & 
		{\texttt{Yes} \\
			\texttt{No} \\
			\texttt{Yes} \\
			\texttt{No}} & 
		{Първата заявка се отнася за подмасива: $13, 15, 13, 5, 1$. Разделяме го на следните множества: $\{13, 5, 1\}$ и $\{13, 15\}$. \textit{Побитовото или} на първото множество е 13, колкото е \textit{побитовото и} на второто множество. \\
			Третата заявка се отнася за подмасива: $13, 5, 1, 7$. Разделяме го на следните множества: $\{5, 1\}$ и $\{13, 7\}$. \textit{Побитовото или} на първото множество е 5, колкото е \textit{побитовото и} на второто множество.} \\
		\hline
	\end{tblr}
\end{table}
\FloatBarrier

\end{document}